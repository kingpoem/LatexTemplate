\section{模型建立}
\section*{无风情况建模}
根据假设条件3、4,建立火势变化率微分方程:
\begin{equation}\label{eq:fire_rate}
    \frac{\mathrm{d}B}{\mathrm{d}t} = 
    \begin{cases}
        \beta t, & 0 \leq t \leq t_1 \\
        \beta t - \lambda X, & t_1 < t < t_2
    \end{cases}
\end{equation}

积分计算总烧毁面积时,几何解释为:
\begin{equation}\label{eq:area_integral}
    B(t_{2}) = \int_{0}^{t_{2}} \frac{\mathrm{d}B}{\mathrm{d}t}\mathrm{d}t = \frac{1}{2}bt_{2}
\end{equation}

\begin{center}
\begin{figure}[H]
    \centering
    \includegraphics[width=0.8\textwidth]{./latex/figure/fire_spread}
    \caption{火势变化率动态示意图}
    \label{fig:fire_spread}
\end{figure}
\end{center}

图示呈现火势变化率$\frac{\mathrm{d}B}{\mathrm{d}t}$在两个阶段的动态特征:
\begin{itemize}[leftmargin=2em]
    \item \textbf{自然蔓延阶段}($t \in [0, t_1]$):火势变化率线性增长,满足$\frac{\mathrm{d}B}{\mathrm{d}t} = \beta t$
    \item \textbf{救援干预阶段}($t \in [t_1, t_2]$):火势变化率线性递减,满足$\frac{\mathrm{d}B}{\mathrm{d}t} = \beta t - \lambda X$
\end{itemize}

记$t = t_{1}$时$\frac{dB}{dt} = b$。烧毁面积$B(t_{2}) = \int_{0}^{t_{2}} \frac{dB}{dt}dt$为图中三角形的面积,有$B(t_{2}) = \frac{1}{2}bt_{2}$。

% 时间关系式
\begin{equation}\label{eq:time_relation}
    t_{2} - t_{1} = \frac{\beta t_{1}}{\lambda^{\theta}X - \beta}
\end{equation}

总烧毁面积公式包含自然蔓延和救援干预两部分贡献:
\begin{equation}\label{eq:total_area}
    B(t_{2}) = \underbrace{\frac{\beta t_{1}^{2}}{2}}_{\text{自然蔓延}} + \underbrace{\frac{\beta^{2} t_{1}^{2}}{2(\lambda^{\theta}X - \beta)}}_{\text{救援减损}}
\end{equation}

救援总费用模型建立时考虑三方面成本:
\begin{equation}\label{eq:cost_model}
    C(X) = \underbrace{c_{1}B(t_2)}_{\text{损失成本}} + \underbrace{\frac{c_{2}\beta t_{1}X}{\lambda^{\theta}X - \beta}}_{\text{时间成本}} + \underbrace{c_{3}X}_{\text{人力成本}}
\end{equation}

\section*{有风情况建模}
当风速不可忽略时,建立运动扩散模型。设圆心运动轨迹:
\begin{equation}\label{eq:fire_center}
    (x - vt)^2 + y^2 = (\beta t)^2
\end{equation}

应用Green定理计算扫掠面积时,需分解为包络线和圆弧两部分:
\begin{equation}\label{eq:green_area}
    A = \frac{1}{2}\oint_{C} (x\,\mathrm{d}y - y\,\mathrm{d}x) = \underbrace{I_1}_{\text{包络线贡献}} + \underbrace{I_2}_{\text{圆弧贡献}}
\end{equation}

最终费用模型体现风速影响因子:
\begin{equation}\label{eq:wind_cost}
    C(X) = c_1(A_1 + A_2 - \pi (\beta t_1)^2) + \frac{c_2 \beta t_1 X}{\lambda^{\theta}X - \beta} + c_3 X
\end{equation}

% ========== 原有内容优化后 ==========
\section*{关键参数关系}
火灾扑灭时间$t_2$与消防力量$X$呈双曲线关系,由式\eqref{eq:time_relation}可知:
\begin{equation}\label{eq:extinguish_time}
    t_2 \propto \frac{1}{\lambda^{\theta}X - \beta}
\end{equation}

最优解表达式揭示临界条件:
\begin{equation}\label{eq:optimal_condition}
    X_{\text{opt}} = \frac{(b+\Delta)\beta}{k} \left( \frac{1}{\lambda} + \sqrt{\frac{c_1 k \lambda t_1^2}{2c_3 (b+\Delta)} + \frac{c_2}{c_3}t_1} \right)
\end{equation}
其中平方根项反映成本权衡机制,当$c_3$(人力成本)增大时,最优人数$X$会相应减少。

\section*{有风情况特殊情形}
当$v > \beta$时,火场呈现拉长椭圆形态,此时:
\begin{equation}\label{eq:envelope_equation}
    y = \pm\frac{\beta}{\sqrt{v^2 - \beta^2}}x
\end{equation}
该包络线方程显示火势前沿的渐近线特性,当$v \gg \beta$时,包络线趋近于直线$y = \pm(\beta/v)x$。

\section{模型求解}
\subsection*{无风情况最优解}
通过极值条件确定最优消防力量:
\begin{equation}\label{eq:windless_optimal}
    \frac{\mathrm{d}C}{\mathrm{d}X} = 0
\end{equation}

解得应派队员人数:
\begin{equation}\label{eq:windless_solution}
    X = \frac{(b + \Delta)\beta}{k} \left( \frac{1}{\lambda} + \sqrt{\frac{c_1 k \lambda t_1^2}{2c_3 (b + \Delta)} + \frac{c_2}{c_3}t_1} \right)
\end{equation}

\subsection*{有风情况分析}
当火势扩散速度$v > \beta$时,解的形式为:
\begin{equation}\label{eq:wind_solution}
    X = \frac{b + \Delta}{k\lambda}\left(\beta + \sqrt{\frac{2c_1\beta t_1\sqrt{v^2 - \delta^2} + c_2\beta^2t_1}{c_3}} \right)
\end{equation}
其中$\delta = \frac{k\lambda}{b+\Delta}X - \beta$表征消防力量对火势的抑制强度。

\section{结论}
\begin{itemize}
    \item \textbf{常规火情}(无风或$v \leq \beta$):
    \begin{equation}\label{eq:general_solution}
        X = \frac{(b+\Delta)\beta}{k}\left(\frac{1}{\lambda} + \sqrt{\frac{c_1k\lambda t_1^2}{2c_3(b + \Delta)} + \frac{c_2}{c_3}t_1}\right)
    \end{equation}

    \item \textbf{强风火情}($v > \beta$):
    \begin{equation}\label{eq:strong_wind_solution}
        X = \frac{b+\Delta}{k\lambda}\left(\beta + \sqrt{\frac{2c_1\beta t_1\sqrt{v^2 - \delta^2} + c_2\beta^2t_1}{c_3}}\right)
    \end{equation}
\end{itemize}
