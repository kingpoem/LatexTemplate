\section{模型建立}

通过分析库存变化规律,建立如下经济订货批量模型:

\begin{itemize}
    \item \textbf{贮存费计算}:
    
    积分分段处理库存变化函数:
    \begin{equation}\label{eq:storage_cost}
        c_2\int_0^T q(t)dt = c_2\left[ \int_0^{T_0} (k - r)t\,dt + \int_{T_0}^T (-rt + rT)dt \right]
    \end{equation}

    \item \textbf{总费用公式推导}:
    
    综合各项成本得周期总费用:
    \begin{equation}\label{eq:total_cost}
        \overline{C} = c_1 + c_2\left( \frac{1}{2}rT^2 - rT_0T + \frac{1}{2}kT_0^2 \right)
    \end{equation}

    \item \textbf{储存量约束关系}:
    
    由生产-需求平衡得:
    \begin{equation}\label{eq:constraint}
        (k - r)T_0 = r(T - T_0) \Rightarrow T = \frac{k}{r}T_0
    \end{equation}
\end{itemize}

最终通过联立式\eqref{eq:constraint}和\eqref{eq:total_cost}可求解最优订货周期$T^*$。

\section{模型求解}

通过极值分析得到最优生产策略:

\begin{itemize}
    \item \textbf{日均费用函数}:
    
    建立含购买成本的目标函数:
    \begin{equation}\label{eq:daily_cost}
        C(T) = \frac{c_1}{T} + \frac{c_2rT}{2} + \theta r
    \end{equation}

    \item \textbf{优化条件分析}:
    
    对生产速率分情形讨论:
    \begin{itemize}
        \item \textbf{情形一} ($k > r$):
        \begin{align}
            T^* &= \sqrt{\frac{2c_1}{c_2r\left(1-\frac{r}{k}\right)}} \label{eq:case1_T} \\
            Q^* &= \sqrt{\frac{2c_1r}{c_2\left(1-\frac{r}{k}\right)}} \label{eq:case1_Q}
        \end{align}
        
        \item \textbf{情形二} ($k = r$):
        \begin{align}
            T^* &= \sqrt{\frac{2c_1}{c_2r}} \label{eq:case2_T} \\
            Q^* &= \sqrt{\frac{2c_1r}{c_2}} \label{eq:case2_Q}
        \end{align}
    \end{itemize}

    \item \textbf{极值推导}:
    
    通过导数法求解式\eqref{eq:daily_cost}:
    \begin{equation}\label{eq:derivative}
        \frac{dC(T)}{dT} = -\frac{c_1}{T^2} + \frac{c_2r}{2} = 0
    \end{equation}
    解得关键结果:
    \begin{equation}\label{eq:optimal_T}
        T^* = \sqrt{\frac{2c_1}{c_2r}}
    \end{equation}
\end{itemize}


最终公式\eqref{eq:optimal_T}与经典EOQ公式具有相同数学形式,表明购买成本$\theta$不改变最优周期决策。
