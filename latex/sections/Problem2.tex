\section{模型求解}

\subsection{参数估计方法}
采用非线性最小二乘法估计参数$K$,$N_0$,$r$。目标函数为:
\begin{equation}
\min_{K,N_0,r}\sum_{i=1}^{n}\left[y_i-\frac{K}{1+\left(\frac{K-N_0}{N_0}\right)e^{-rt_i}}\right]^2
\end{equation}

展开Logistic函数表达式:
\begin{equation}
N(t) = \frac{K}{1 + \left(\frac{K - N_0}{N_0}\right)e^{-rt}}
\end{equation}

建立残差函数:
\begin{equation}
r_i = y_i - N(t_i),\quad i=1,2,\cdots,n
\end{equation}

采用Levenberg-Marquardt算法进行数值优化,迭代公式:
\begin{equation}
\boldsymbol{\theta}^{(k+1)} = \boldsymbol{\theta}^{(k)} - (\mathbf{J}^\top\mathbf{J} + \lambda \mathbf{I})^{-1}\mathbf{J}^\top\mathbf{r}
\end{equation}
其中$\boldsymbol{\theta} = (K, N_0, r)^\top$为待估参数,$\mathbf{J}$为雅可比矩阵,$\lambda$为阻尼因子。

\subsection{数值计算步骤}
\begin{enumerate}
\item 初始化参数:设$K^{(0)} = 15$亿,$N_0^{(0)} = 5$亿,$r^{(0)} = 0.05$
\item 计算初始残差:$\mathbf{r}^{(0)} = [y_i - N(t_i;\boldsymbol{\theta}^{(0)})]$
\item 迭代求解:
\begin{itemize}
\item 最大迭代次数:100次
\item 收敛标准:$\|\mathbf{r}^{(k)}\|_2 < 10^{-6}$
\end{itemize}
\item 计算标准误差:
\begin{equation}
SE(\hat{\theta}_j) = \sqrt{\hat{\sigma}^2 [(\mathbf{J}^\top\mathbf{J})^{-1}]_{jj}}
\end{equation}
\end{enumerate}

\subsection{求解结果}
通过Python编程实现得到以下参数估计值:
\begin{table}[htbp]
\centering
\caption{Logistic模型参数估计结果}
\begin{tabular}{ccc}
\toprule
\textbf{环境容纳量K} & \textbf{初始人口$N_0$} & \textbf{增长率r} \\
\midrule
15.7585亿人 &5.1358亿人 &0.0410/年 \\
\bottomrule
\end{tabular}
\end{table}

\subsection{拟合效果评估}
\begin{figure}[htbp]
\centering
\includegraphics[width=0.8\textwidth]{./latex/figure/population}
\caption{中国人口数据与Logistic模型拟合对比}
\end{figure}

拟合优度指标:
\begin{itemize}
\item $R^2=0.9976$
\item RMSE=0.1194百万人
\end{itemize}
