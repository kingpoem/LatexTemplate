\section{模型假设}\label{sec:Premise}

在森林火灾救援问题中,我们做出以下基本假设:

\begin{itemize}[leftmargin=2em]
    \item \textbf{假设1}(损失费用):损失费用$L$与最终烧毁面积$B(t_2)$成正比,即
        \begin{equation}
            L = c_1 B(t_2)
        \end{equation}
        其中$c_1$为烧毁单位面积的损失费用系数。

    \item \textbf{假设2}(灭火速度):消防队员灭火速度$\lambda$与初始火势$b$满足
        \begin{equation}
            \lambda(\theta) = \frac{k}{b + \Delta}
        \end{equation}
        其中$k$为灭火能力系数,$\Delta$为防止分母过小的修正量,$\theta$为火势方向参数。

    \item \textbf{假设3}(火势蔓延):无风情况下,$t \in [0, t_1]$时火势蔓延速率满足
        \begin{equation}
            \frac{dB}{dt} = \beta t
        \end{equation}
        其中$\beta$为火势蔓延基础速率,该假设源于圆形蔓延模型$B(t) \propto t^2$。

    \item \textbf{假设4}(救援效果):派出$x$名消防队员后,$t \in [t_1, t_2]$时火势蔓延速率修正为
        \begin{equation}
            \frac{dB}{dt} = \beta t - \lambda x
        \end{equation}
        其中$\lambda$需满足$\lambda x > \beta t_1$以保证灭火可行性。

    \item \textbf{假设5}(费用结构):总救援费用包含时间相关费用和固定支出
        \begin{equation}
            C = c_2 x (t_2 - t_1) + c_3 x
        \end{equation}
        其中$c_2$为单位时间人力成本,$c_3$为人均固定成本。

    \item \textbf{假设6}(几何蔓延):无风时烧毁面积满足
        \begin{equation}
            B(t) = \pi (v t)^2 \implies \frac{dB}{dt} = 2\pi v^2 t
        \end{equation}
        其中$v$为火势径向蔓延速度,该模型在风速小于3m/s时成立。

    \item \textbf{假设7}(风场影响):当风速$w \geq 5$m/s时,蔓延模型修正为
        \begin{equation}
            \frac{dB}{dt} = \beta t \cdot (1 + \gamma w^2)
        \end{equation}
        其中$\gamma$为风场耦合系数,该修正项仅在主风方向与救援通道夹角$<45^\circ$时适用。
\end{itemize}
