\section{模型假设}\label{sec:Premise}

在建模过程中,我们做出以下基本假设:

% \begin{itemize}
%     \item \textbf{假设1}:汽车通过路障时的速度为零
    
%     该假设确保每个路障都能有效重置车辆运动状态,符合强制减速的设计目标。

%     \item \textbf{假设2}:汽车通过路障后做等加速运动
    
%     基于车辆动力系统的物理特性,假设恒定加速度便于建立运动学方程。

%     \item \textbf{假设3}:汽车达到限速$v_{\max}$时立即开始减速运动
    
%     该假设保证车辆在达到最高允许速度后立即进入制动阶段,符合安全驾驶规范。

% \end{itemize}

\begin{itemize}
\item \textbf{有限资源假设}:假设中国国土资源和环境条件能够支持的人口数量存在上限$K$;
\item \textbf{增长率递减假设}:随着人口接近环境容纳量,人口增长率逐渐降低;
\item \textbf{封闭系统假设}:暂不考虑国际迁移对人口的显著影响;
\item \textbf{政策连续性假设}:假设未来人口政策保持相对稳定;
\item \textbf{无重大突发事件}:假设不会发生大规模战争、瘟疫等突发事件。
\end{itemize}