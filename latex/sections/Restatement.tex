\section{问题重述}
学生需在满足课程先修关系约束下制定最优选课方案。设共有9门候选课程,每门课程具有学分、学科属性和前置知识要求(见表1)。需建立数学模型解决以下两类优化问题:

\begin{itemize}
    \item \textbf{最少课程覆盖问题}:选择满足所有先修条件的最小课程集合,确保能修完所选全部课程
    \item \textbf{多目标优化问题}:在课程数量最少的基础上,最大化总学分收益,寻求帕累托最优解
\end{itemize}

需特别注意课程间的复杂依赖关系,如:
\begin{itemize}
    \item 基础课程(微积分、线性代数)作为多门前沿课程的共同先修
    \item 编程类课程(计算机编程)是计算机与运筹学课程的共同基础
    \item 存在跨学科课程的复合先修要求(如数学实验需同时具备微积分和线性代数基础)
\end{itemize}

\begin{table}[htbp]
\centering
\caption{课程属性表}
\label{tab:course}
\begin{tabular}{ccccc}
\toprule
编号 & 课程名称 & 学分 & 类别 & 先修要求 \\
\midrule
1 & 微积分 & 5 & 数学 & 无 \\
2 & 线性代数 & 4 & 数学 & 无 \\
3 & 最小化方法 & 4 & 数/运 & 1,2 \\
4 & 数据结构 & 3 & 数/计 & 7 \\
5 & 应用统计 & 4 & 数/运 & 1,2 \\
6 & 计算机模拟 & 3 & 计/运 & 7 \\
7 & 计算机编程 & 2 & 计算机 & 无 \\
8 & 预测理论 & 2 & 运筹学 & 5 \\
9 & 数学实验 & 3 & 计/运 & 1,2 \\
\bottomrule
\end{tabular}
\end{table}
\section{符号说明}
\begin{table}[!ht]
\centering
\caption{符号定义表}
\label{tab:symbols}
\begin{tabular}{cl}
\toprule
符号 & 说明 \\
\midrule
$x_i$ & 课程选择决策变量,$x_i \in \{0,1\},\ i=1,\dots,9$ \\
$C_i$ & 课程$i$的学分值(具体数值见表\ref{tab:course}) \\
$\mathcal{P}_j$ & 课程$j$的前驱课程集合 \\
$w_1$ & 课程数量目标的权重系数 $(w_1 > 0)$ \\
$w_2$ & 学分目标的权重系数 $(w_2 > 0)$ \\
\bottomrule
\end{tabular}
\end{table}

\section{模型建立}
\subsection{多目标规划模型}
建立如下双目标优化问题:
\begin{equation}
\begin{cases}
\displaystyle \min\ f_1 = \sum_{i=1}^{9} x_i \\
\displaystyle \max\ f_2 = \sum_{i=1}^{9} C_i x_i 
\end{cases}
\end{equation}

\subsection{单目标转化}
通过线性加权法构造综合目标函数:
\begin{equation}
\min\ Z = w_1 \sum_{i=1}^{9} x_i - w_2 \sum_{i=1}^{9} C_i x_i
\end{equation}

\subsection{约束条件}
\begin{align}
x_3 & \leq \frac{1}{2}(x_1 + x_2) \quad (\text{最小化方法}) \\
x_4 & \leq x_7 \quad (\text{数据结构}) \\
x_5 & \leq \frac{1}{2}(x_1 + x_2) \quad (\text{应用统计}) \\
x_6 & \leq x_7 \quad (\text{计算机模拟}) \\
x_8 & \leq x_5 \quad (\text{预测理论}) \\
x_9 & \leq \frac{1}{2}(x_1 + x_2) \quad (\text{数学实验}) \\
x_i & \in \{0,1\},\ i=1,\dots,9
\end{align}

\section{模型求解}
采用分支定界法求解0-1整数规划问题,具体步骤:
\begin{enumerate}
\item 标准化处理:将多目标转化为单目标
\item 松弛处理:暂时忽略整数约束,求解线性规划问题
\item 分支操作:对非整数值变量进行分支搜索
\item 剪枝策略:利用目标函数值进行剪枝优化
\end{enumerate}

\section{结论分析}
通过Python求解得到:
\begin{itemize}
\item \textbf{问题1}:最少需选6门课程(课程1,2,3,5,7,9)
\item \textbf{问题2}:当$w_1:w_2=1:2$时,最优方案总学分达21分(课程1,2,3,5,7,9)
\item 关键路径分析显示:微积分和线性代数是核心前置课程
\end{itemize}
