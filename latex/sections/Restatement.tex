\section{问题分析}

存贮模型问题可抽象为典型的库存优化问题,其核心在于建立经济订货批量模型(EOQ Model)来确定最优存贮策略。考虑以下要素:

\begin{itemize}
    \item \textbf{系统场景}:涵盖工厂原料采购、车间零件加工、商品零售库存、水库蓄水调度等典型存贮场景
    
    \item \textbf{决策变量}:
        \begin{itemize}
            \item 订货周期$T$(单位:天)
            \item 订货批量$Q$(单位:件)
        \end{itemize}
    
    \item \textbf{成本构成}:
        \begin{itemize}
            \item 存贮成本:与库存量成正比的持有成本
            \item 订购成本:每次订货产生的固定成本
            \item 货物成本:新增考虑的采购成本(单价成本)
        \end{itemize}
    
    \item \textbf{优化目标}:最小化单位时间总成本
        \[
        \min_{T,Q} C(T,Q) = \frac{c_1}{T} + \frac{1}{2}c_2 Q T + c_3 Q
        \]
        其中$c_1$为准备费,$c_2$为贮存费率,$c_3$为货物单价
    
    \item \textbf{约束条件}:
        \begin{itemize}
            \item 需求约束:$Q = rT$($r$为日需求率)
            \item 非负约束:$T>0, Q>0$
            \item 存量约束:考虑两种情形(不允许缺货/允许缺货)
        \end{itemize}
\end{itemize}

该问题的扩展性在于:相较于经典EOQ模型,新增货物成本项后,需重新推导最优解$(T^*, Q^*)$,并通过比较静态分析探讨成本结构变化对最优策略的影响。
