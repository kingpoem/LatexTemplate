\section{问题重述}
当研究国家/地区人口随时间变化规律时,需建立数学模型描述其增长趋势。基于1790-2000年美国人口数据,要求:
\begin{itemize}
    \item 建立指数增长模型并进行参数估计
    \item 验证模型对短期预测的适用性
    \item 量化模型与实际数据的偏差
\end{itemize}

\section{问题假设}\label{sec:Premise}
\begin{itemize}[leftmargin=2em]
    \item \textbf{恒定增长率}:单位时间人口增长率$r$为常数,满足
        \begin{equation}
            \frac{\mathrm{d}x(t)}{\mathrm{d}t} = r x(t)
        \end{equation}
    \item \textbf{初始条件}:设初始时刻$t=0$(对应1790年)的人口规模为
        \begin{equation}
            x(0) = x_0
        \end{equation}
\end{itemize}

\section{模型建立}\label{sec:Model}
人口指数增长模型的微分方程及其解析解:
\begin{equation}\label{eq:diff_eq}
    \left\{
    \begin{aligned}
        &\frac{dx}{dt} = r x, \\
        &x(0) = x_0
    \end{aligned}
    \right.
\end{equation}
求解得指数增长模型:
\begin{equation}
    x(t) = x_0 e^{r t} \label{eq:exp_growth}
\end{equation}
其中$t$以年为单位,$r>0$时为增长速率常数。

\section{模型求解}
采用线性最小二乘法进行参数估计,具体步骤:
\begin{enumerate}
    \item \textbf{数据预处理}:将原始数据转换为相对时间$t = (year - 1790)$,人口数据取自然对数$y = \ln x$
    
    \item \textbf{建立线性模型}:
    \begin{equation}
        y = r t + \ln x_0 \label{eq:linear_model}
    \end{equation}

    \item \textbf{构建设计矩阵}:对$n$个数据点,构造
    \begin{equation}
        \mathbf{A} = \begin{bmatrix}
            t_1 & 1 \\
            t_2 & 1 \\
            \vdots & \vdots \\
            t_n & 1
        \end{bmatrix}, \quad
        \mathbf{y} = \begin{bmatrix}
            y_1 \\ y_2 \\ \vdots \\ y_n
        \end{bmatrix}
    \end{equation}

    \item \textbf{求解正规方程}:通过矩阵运算得到参数估计
    \begin{equation}
        \begin{bmatrix}
            r \\ \ln x_0
        \end{bmatrix} = (\mathbf{A}^T \mathbf{A})^{-1} \mathbf{A}^T \mathbf{y}
    \end{equation}

    \item \textbf{参数转换}:计算初始人口$x_0 = \exp(\ln x_0)$
\end{enumerate}

\section{结论}
\begin{figure}[H]
    \centering
    \includegraphics[width=0.8\textwidth]{./latex/figure/fitting}
    \caption{population growth model fitting}
    \label{fig:conclusion}
\end{figure}

模型最终求解结果:
\begin{equation}
    \left\{
    \begin{aligned}
        x_0 &= 6.0496 \\
        r &= 0.2020\ \text{(每十年)}
    \end{aligned}
    \right.
\end{equation}